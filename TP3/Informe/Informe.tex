\documentclass[12pt, titlepage]{article}
\usepackage{prftree}
\usepackage{amssymb}
\usepackage{amsmath}
\usepackage{enumitem} 


\title{Trabajo Práctico 3\\ 
Análisis de Lenguajes de Programación}
\author{Agustín Fernández Bergé y Ramiro Gatto\\ Legajos: F-3726/5 y G-5803/3}
\date{03/12/2024}

\begin{document}
\maketitle

\section*{Ejercicio 1.a}
\noindent{Dada la siguienete monada State:}

\noindent $instance$ $Monad$ $State$ $where$

\hspace{15pt}$return$ $x = State (\lambda s \rightarrow (x$ :!: $s))$

\hspace{15pt}$m >>= f = State (\lambda s \rightarrow let$ $(v$ :!: $s') = runState$ $m$ $s$

\hspace{145pt}$in$ $runState (f$ $v)$ $s')$

\noindent{Hay qur probamos que en efecto es una monada, para esto debemos probar las tres leyes de monada.}

\subsection*{Monad.1}
Debemos probamos que: $return$ $s >>= k = k$ $a$\\
Demostración:\\
$return$ $a >>= k$\\
$= <def.$ $return >$\\
$State (\lambda s \rightarrow  (a$ :!: $s)) >>= k$ \\
$= < def. >>= >$\\
$State (\lambda s \rightarrow$ 
$let$ $(v$ :!: $s') = runState (State (\lambda s'' \rightarrow (a $ :!: $ s'')))$ $s$

\hspace{45pt} $in$ $runState$ $(k$ $v)$ $s')$\\
$=< def.$ $runState.State = id >$ \\
$State (\lambda s \rightarrow let$ $(v$ :!: $s') = (\lambda s'' \rightarrow  (a$ :!: $s'')) s$

\hspace{45pt} $in$ $runState (k$ $v)$ $s')$\\
$= < def.$ $App. >$\\
$State (\lambda s \rightarrow let$ $(v$ :!: $s') = (a :!: s)$

\hspace{45pt} $in$ $runState (k$ $v)$ $s')$\\
$= <a=v$ y $a=s'>$\\
$State (\lambda s \rightarrow let$ $(v$ :!: $s') = (a :!: s)$

\hspace{45pt} $in$ $runState (k$ $a)$ $s)$\\
$=<def. let>$\\
$State (\lambda s \rightarrow runState (k$ $a)$ $s)$\\
$= <def.$ $App.>$\\
$State(runState (k$ $a))$\\
$= <def.$ $runState.State = id>$\\
$k$ $a$

%%===========================
\newpage
\subsection*{Monad.2}
Debemos probamos que: $m >>= return = m$\\
Demostración:\\
$m >>= return$\\
$= <def. >>=>$\\
$State (\lambda s \rightarrow  let$ $(v$ :!: $s') = runState$ $m$ $s$

\hspace{45pt} $in$ $runState (return$ $v)$ $s')$\\
$=<def.$ $return>$\\
$State (\lambda s \rightarrow  let$ $(v$ :!: $s') = runState$ $m$ $s$

\hspace{45pt} $in$ $runState (State (\lambda s'' \rightarrow(v$ :!: $s'')))$ $s')$\\
$=<def.$ $runState.State = id>$\\
$State (\lambda s \rightarrow  let$ $(v$ :!: $s') = runState$ $m$ $s$
 	    
\hspace{45pt}$in$ $(\lambda s'' \rightarrow (v$ :!: $s''))$ $s')$\\
$=<def.$ $App.>$\\
$State (\lambda s \rightarrow  let$$(v$ :!: $s') = runState$ $m$ $s$
 	   
\hspace{43pt} $in$ $(v$ :!: $s'))$\\
$=<Prop.$ $let>$\\
$State (\lambda s \rightarrow  runState$ $m$ $s)$\\
$=<def.$ $App.>$\\
$State (runState$ $m)$\\
$=<def.$ $runState.State = id>$\\
$m$

%=============================
\newpage
\subsection*{Monad.3}
Debemos probar que: $(m >>= k) >>= h = m >>= (\lambda x \rightarrow k x >>= h )$\\
Demostración:\\

\noindent I) Primera mitad:\\
$(m >>= k) >>= h$\\
$=<def. >>=>$\\
$State (\lambda s1 \rightarrow  let$ $(v1$ :!: $s1') = runState$ $m$ $s1$
 	    
\hspace{45pt} $in$ $runState (k$ $v1)$ $s1') >>= h$\\
$=<def. >>=>$\\
$State (\lambda s$$\rightarrow$$let$ $(v$:!:$s')$$=
$$runState(State (\lambda s1$$\rightarrow$$let$ $(v1$:!:$s1') = runState$ $m$ $s1$

\hspace{210pt} $in$ $runState (k$ $v1)$ $s1'))$ $s$

\hspace{40pt}$in$ $runState (h$ $v)$ $s')$\\
$=<def.$ $runState.State = id>$\\
$State (\lambda s$$\rightarrow$$let$ $(v$:!:$s')$$=
$$(\lambda s1$$\rightarrow$$let$ $(v1$:!:$s1') = runState$ $m$ $s1$

\hspace{130pt} $in$ $runState (k$ $v1)$ $s1')$ $s$

\hspace{40pt}$in$ $runState (h$ $v)$ $s')$\\
$=<def.$ $App.>$\\
$State (\lambda s \rightarrow let$ $(v :!: s') = (let (v1 :!: s1') = runState$ $m$ $s$

\hspace{215pt}$in$ $runState (k$ $v1)$ $s1')$

\hspace{48pt}$in$ $runState (h$ $v)$ $s')$\\
$=<Prop.$ $Let>$\\
$State (\lambda s \rightarrow let$ $(v1$ :!: $s1') = runState$ $m$ $s$

\hspace{62pt}$(v :!: s') = runState (k$ $v1)$ $s1'$

\hspace{45pt}$in$ $runState (h$ $v)$ $s')$\\


\noindent II) Segunda mitad:\\
$m >>= (\lambda x \rightarrow k$ $x >>= h )$\\
$=<def.$ $>>=>$\\
$State (\lambda s \rightarrow let$ $(v1$ :!: $s1') = runState$ $m$ $s$

\hspace{45pt} $in$ $runState ((\lambda x \rightarrow k x >>= h ) v) s')$\\
$=<def.$ $App>$\\
$State (\lambda s \rightarrow let (v1$ :!: $s1') = runState$ $m$ $s$

\hspace{40pt}$in$ $runState (k$ $v >>= h )$ $s')$\\
$=<def.$ $>>=>$\\
\newpage
\noindent $State (\lambda s \rightarrow let (v1$ :!: $s1') = runState$ $m$ $s$

\hspace{43pt}$in$ $runState ( State (\lambda s'' \rightarrow let$ $(v$ :!: $s') = runState$ $(k$ $v)$ $s''$

\hspace{175pt} $in$ $runState (h$ $v)$ $s')$

\hspace{101pt} $)$ $s1')$\\
$=<def.$ $runState.State = id>$\\
$State (\lambda s \rightarrow let$ $(v1$ :!: $s1') = runState$ $m$ $s$

\hspace{45pt}$in (\lambda s'' \rightarrow let$ $(v$ :!: $s') = runState$ $(k$ $v)$ $s''$

\hspace{92pt} $in$ $runState$ $(h$ $v)$ $s')$

\hspace{50pt} $s1')$\\
$=<def.$ $App.>$\\
$State (\lambda s \rightarrow let$ $(v1$ :!: $s1') = runState$ $m$ $s$

\hspace{43pt} $in$ $let$ $(v$ $:!:$ $s') = runState$ $(k$ $v)$ $s1'$

\hspace{56pt} $in$ $runState$ $(h$ $v)$ $s')$\\

$=<def.$ $Let>$\\
$State (\lambda s \rightarrow let$ $(v1$ :!: $s1') = runState$ $m$ $s$

\hspace{63pt}$(v$ :!: $s') = runState$ $(k$ $v)$ $s1$

\hspace{47pt}$in$ $runState$ $(h$ $v)$ s'$)$\\

\noindent De I y II se tiene que vale: $(m >>= k) >>= h = m >>= (\lambda x \rightarrow k x >>= h )$\

\subsection*{Conclusion}
Como se cumple, monad.1, monad.2 y monad.3, tenemos que en efecto es una monada
\end{document}