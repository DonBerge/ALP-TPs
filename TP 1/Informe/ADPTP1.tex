\documentclass[12pt, titlepage]{article}
\usepackage[utf8]{inputenc}
\usepackage{prftree}
\usepackage{amssymb}
\usepackage{amsmath}
\usepackage{enumitem} 

\usepackage[spanish]{babel}
\usepackage{amsthm}
\usepackage{bussproofs}
\usepackage{bm}

\renewcommand{\skip}{\textbf{skip}}
\newcommand{\true}{\textbf{true}}
\newcommand{\false}{\textbf{false}}
\newcommand{\until}{\textbf{until}}

\newcommand{\lbangle}{\bm{\langle}}
\newcommand{\rbangle}{\bm{\rangle}}
\newcommand{\deriva}{\rightsquigarrow}
\newcommand{\bexp}{\Downarrow_{exp}}
\newcommand{\pp}{\texttt{++}}
\newcommand{\mm}{\texttt{--}}

\title{Trabajo Práctico 1\\ 
Análisis de Lenguajes de Programación}
\author{Agustin Fernandez y Ramiro Gatto}
\date{17/9/2024}

\begin{document}
\maketitle

\section{Ejercicio 1}
Para este ejercicio lo que se hizo fue modificar parte de las sintaxis 
abstracta y concreta, las partes que se modificaron fueron unicamente el 
intexp en ambos, quedando.\\

\noindent \textbf{Sintaxis Abstracta}

$intexp ::= nat \: | \: var | \: -_u intexp$

\hspace{1.5cm}$|\: intexp +intexp$

\hspace{1.5cm}$|\: var \mathrel{+{+}}$

\hspace{1.5cm}$|\: var \mathrel{-{-}}$

\hspace{1.5cm}$|\: intexp -_b intexp$

\hspace{1.5cm}$|\: intexp \times intexp$

\hspace{1.5cm}$|\: intexp \div  intexp $\\

\noindent \textbf{Sintaxis Concreta}

$intexp ::= nat \:|\: var$

\hspace{1.5cm}$|\: var $ '$++$'

\hspace{1.5cm}$|\: var $ '$--$'

\hspace{1.5cm}$|\: var$

\hspace{1.5cm}$|\:$ '$-$' $intexp$

\hspace{1.5cm}$|\: intexp$ '$+$' $intexp$

\hspace{1.5cm}$|\: intexp$ '$-$' $intexp$

\hspace{1.5cm}$|\: intexp$ '$*$' $intexp$

\hspace{1.5cm}$|\: intexp$ '$/$' $intexp$

\hspace{1.5cm}$|\: $'('$ intexp $')'


\section{Ejercicio 4}
\begin{displaymath}   
    \prftree[r]{VarInc}{x \in \operatorname{dom} \sigma} {\langle x \pp,\sigma \rangle \Downarrow_{exp} \langle \sigma x \bm{+} \bm{1}, [\sigma| x : \sigma x \bm{+} \bm{1}] \rangle}
\end{displaymath}

\begin{displaymath}   
    \prftree[r]{VarDec}{x \in \operatorname{dom} \sigma} {\langle x \mm,\sigma \rangle \Downarrow_{exp} \langle \sigma x \bm{-} \bm{1}, [\sigma| x : \sigma x \bm{-} \bm{1}] \rangle}
\end{displaymath}


\section{Ejercicio 5}
Probamos que la relación $\leadsto$ es determinista, para ello usamos Inducción en la derivación t $\leadsto$ t'.\\
Se tiene que la propiedad a probar seria:  

\begin{equation*}
\begin{aligned}
    t \leadsto t'\\
    t \leadsto t''
\end{aligned}
\quad \Rightarrow \quad
\begin{aligned}
    t' = t''
\end{aligned}
\end{equation*}


\begin{itemize}[label=$\bullet$]

\item {Caso Base (Ultima regla utilizada Ass)}
    \\Sabemos que:
    \begin{itemize}[label=$\bullet$]
        \item {$\langle e, \sigma \rangle \Downarrow_{exp} \langle n, \sigma' \rangle$}
        \item {$t = \langle v=e, \sigma \rangle$}
        \item {$t' = \langle \skip, [\sigma' | v : n] \rangle$}
    \end{itemize}
    Entonces como solo se le puede aplicar a $t$ Ass y por ser $\Downarrow_{exp}$ determinista se tiene
    que $t' = t''$

\item {Caso Base (Ultima regla utilizada Seq1)}
    \\Sabemos que:
    \begin{itemize}[label=$\bullet$]
        \item {$t = \langle c_0;c_1, \sigma \rangle$}
        \item {$t' = \langle c_0';c_1, \sigma  \rangle$}
    \end{itemize}
    Entonces solo se pueden usar dos reglas, Seq1 y Seq2; pero para poder usar
    Seq2 debe haber regla que verifique: $\langle Skip,\sigma \rangle \leadsto \langle C_0',\sigma' \rangle$.\\
    Pero dicha regla no existe, lo que implica que solo se puede aplicar Seq1\\
    $\therefore$ Como $\Downarrow_{exp} $ determinista, $t' = t''$

\item {Caso Base (Ultima regla utilizada If1)}
    \\Sabemos que:
    \begin{itemize}[label=$\bullet$]
        \item {$\langle b, \sigma \rangle \Downarrow_{exp} \langle \true, \sigma' \rangle$}
        \item {$t = \langle \textbf{if} \:b \:\textbf{then} \:c_0 \:\textbf{else} \:c_1, \sigma \rangle$}
        \item {$t' = \langle c_0, \sigma  \rangle$}
    \end{itemize}
    Entonces solo se pude usar dos reglas para obtener $t''$, If1 e If2; pero si se usara If2
    se tendría que: $\langle b, \sigma \rangle \Downarrow_{exp} \langle \false, \sigma' \rangle$.
    Pero esto contradecir que $\Downarrow_{exp}$ determinista.
    \\Por lo tanto solo se pudo haber usado If1, pero como $\Downarrow_{exp}$ determinista se tiene que
    $t' = t''$     
         
\item {Caso Base (Ultima regla utilizada If2)}
    \\Análogo al caso If1

\item {Caso Base (Ultima regla utilizada Repeat)}
    \\Sabemos que:
    \begin{itemize}[label=$\bullet$]
        \item {$t = \langle \textbf{repeat} \:c \:\until \:b, \sigma \rangle$}
        \item {$t' = \langle c; \textbf{if} \:b \:\textbf{then} \:\skip \:\textbf{else} \:\textbf{repeat} \:c \:\until \:b, \sigma\rangle$}
    \end{itemize}
    Tenemos que solo se pude aplicar una regla a t, la cual es repeat
    \\$\therefore$ Se concluye que $t'= t''$

\item {Caso Inductivo (Ultima regla utilizada Seq2)}
    \\Sabemos que:
    \begin{itemize}[label=$\bullet$]
        \item {$\langle c_0, \sigma \rangle \leadsto \langle c_0', \sigma \rangle$}
        \item {$t = \langle c_0;c_1, \sigma \rangle $}
        \item {$t' = \langle c_0';c_1, \sigma  \rangle$}
    \end{itemize}

    Tenemos entonces dos casos posibles
    \begin{itemize}[label=$\bullet$]
        \item {Se utilizo Seq1}
            \begin{itemize}[label=$\bullet$]
                \item {$t'' =\langle c_1, \sigma \rangle$}
                \item {$t = \langle \skip, \sigma \rangle$}
            \end{itemize}
                Pero esto seria un absurdo, ya que se tendría que t tiene dos formas distintas.
                $\therefore Abs!!!$

        \item {Se utilizo Seq2}
            \begin{itemize}[label=$\bullet$]
                \item {$t'' =\langle k_0;c_1, \sigma'' \rangle$}
            \end{itemize}
            Por la por la forma de Seq2 se tiene que
            \\$\langle c_0, \sigma \rangle \leadsto \langle k_0, \sigma'' \rangle$
            \\Luego por la HI se tiene,
            $\langle c_0, \sigma' \rangle = \langle k_0, \sigma'' \rangle$
            \\$\therefore t'=t''$
    \end{itemize}
\end{itemize}


\section{Ejercicio 6}

Quiero probar que estos programas son semánticamente equivalentes:

\ 

\begin{minipage}{0.45\textwidth}
\begin{flushleft}
a) \\
\texttt{x=x+1} \\
\texttt{y=x}
\end{flushleft}
\end{minipage}
\begin{minipage}{0.45\textwidth}
\begin{flushleft}
b) \\
\texttt{y=x++}
\end{flushleft}
\end{minipage}

\ 

Se asume que $\sigma$ es un estado tal que $x \in \operatorname{dom} \sigma$.

\ 

a)

\begin{prooftree}
    \AxiomC{$x \in \operatorname{dom} \sigma$}
    \RightLabel{V\scriptsize{AR}}
    \UnaryInfC{$\lbangle x, \sigma \rbangle \bexp \lbangle \sigma \  x, \sigma \rbangle$}
    \AxiomC{}
    \RightLabel{NV\scriptsize{AL}}
    \UnaryInfC{$\lbangle 1, \sigma \rbangle \bexp \lbangle \bm{1}, \sigma \rbangle$}
    \RightLabel{P\scriptsize{LUS}}
    \BinaryInfC{$\lbangle x+1, \sigma \rbangle \bexp \lbangle \sigma \ x \bm{+} \bm{1}, \sigma \rbangle$}
    \RightLabel{A\scriptsize{SS}}
    \UnaryInfC{$\lbangle x=x+1 , \sigma \rbangle \deriva \lbangle \skip, [\sigma \ | \  x:\sigma \  x \bm{+} \bm{1}] \rbangle$}
    
    \RightLabel{S\scriptsize{EQ\tiny 2}}
    \UnaryInfC{$\lbangle x=x+1;y=x, \sigma \rbangle \deriva \lbangle \skip;y=x, [\sigma \ | \  x:\sigma \  x \bm{+} \bm{1}] \rbangle$}
\end{prooftree}

\begin{prooftree}
    \AxiomC{}
    \RightLabel{S\scriptsize{EQ\tiny 1}}
    \UnaryInfC{$\lbangle \skip;y=x, [\sigma \ | \  x:\sigma \  x \bm{+} \bm{1}] \rbangle \deriva \lbangle y=x, [\sigma \ | \  x:\sigma \  x \bm{+} \bm{1}] \rbangle$}
\end{prooftree}

\begin{prooftree}
    \AxiomC{$x \in \operatorname{dom} x$}
    \RightLabel{Def $[\sigma | v:e]$}
    \UnaryInfC{$x \in \operatorname{dom} [\sigma \ | \  x:\sigma \  x \bm{+} \bm{1}]$}
    \RightLabel{V\scriptsize{AR}}
    \UnaryInfC{$\lbangle x, [\sigma \ | \  x:\sigma \  x \bm{+} \bm{1}]\rbangle \bexp \lbangle \sigma x \bm{+} \bm{1}, [\sigma \ | \  x:\sigma \  x \bm{+} \bm{1}]\rbangle$}
    \RightLabel{A\scriptsize{SS}}
    \UnaryInfC{$\lbangle y = x, [\sigma \ | \  x:\sigma \  x \bm{+} \bm{1}]\rbangle \deriva \lbangle \skip, [\sigma \ | \  y: \sigma \  x \bm{+} \bm{1}, x:\sigma \  x \bm{+} \bm{1}]\rbangle$}
\end{prooftree}

\ 

Se tiene entonces que:

$\lbangle x=x+1;y=x, \sigma \rbangle$ 

$\deriva$

$\lbangle \skip;y=x, [\sigma \ | \  x:\sigma \  x \bm{+} \bm{1}] \rbangle$

$\deriva$

$\lbangle y=x, [\sigma \ | \  x:\sigma \  x \bm{+} \bm{1}]$

$\deriva$

$\lbangle \skip, [\sigma \ | \  y: \sigma \  x \bm{+} \bm{1}, x:\sigma \  x \bm{+} \bm{1}]\rbangle$

\ 

Por lo tanto: $\lbangle x=x+1;y=x, \sigma \rbangle \deriva^{*} \lbangle \skip, [\sigma \ | \  y: \sigma \  x \bm{+} \bm{1}, x:\sigma \  x \bm{+} \bm{1}]\rbangle$

\ 

b)

Se tiene la siguiente derivación para $\pp$

\begin{prooftree}
    \AxiomC{$x \in \operatorname{dom} \sigma$}
    \RightLabel{V\scriptsize{AR}\small{I}\scriptsize{NC}}
    \UnaryInfC{$\lbangle x\pp, \sigma \rbangle \bexp \lbangle \sigma \ x\bm{+}\bm{1}, [\sigma \ | \ x:\sigma \ x \bm{+} \bm{1}] \rbangle$}
\end{prooftree}

Por lo tanto:
\begin{prooftree}
    
    \AxiomC{$x \in \operatorname{dom} \sigma$}
    \RightLabel{V\scriptsize{AR}\small{I}\scriptsize{NC}}
    \UnaryInfC{$\lbangle x\pp, \sigma \rbangle \bexp \lbangle \sigma \ x\bm{+}\bm{1}, [\sigma \ | \ x:\sigma \ x \bm{+} \bm{1}] \rbangle$}
    \RightLabel{A\scriptsize{SS}}
    \UnaryInfC{$\lbangle y=x\pp, \sigma \rbangle \deriva \lbangle \skip, [\sigma \ | \ y: \sigma \ x \bm{+} \bm{1}, \: x:\sigma \ x \bm{+} \bm{1}] \rbangle$}
\end{prooftree}

Como

$\lbangle x=x+1;y=x, \sigma \rbangle \deriva^{*} \lbangle \skip, [\sigma \ | \  y: \sigma \  x \bm{+} \bm{1}, x:\sigma \  x \bm{+} \bm{1}]\rbangle$

\hspace{0.5cm}y

$\lbangle y=x\pp, \sigma \rbangle \deriva^{*} \lbangle \skip, [\sigma \ | \ y: \sigma \ x \bm{+} \bm{1}, \: x:\sigma \ x \bm{+} \bm{1}] \rbangle$

Se tiene que ambos programas son semánticamente equivalentes.



\end{document}